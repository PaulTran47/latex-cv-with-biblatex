%=================================================
%% 2 percentage signs represent comments for code;
% 1 percentage sign represents commented out code

%% LaTeX-style CV with biblatex package usage
%% Paul Le Tran
%% Last updated: 25 December, 2023.
%=================================================

%=================================================
\documentclass[a4paper, 10pt]{article}

%% Used packages
\usepackage{tran_paul_le_cv, booktabs, fontawesome}
\usepackage[T1]{fontenc}
\usepackage{hyperref}
\usepackage[sfdefault, lf, t]{carlito}
\usepackage{inconsolata}
\usepackage{fmtcount}
\usepackage{refcount}

%% Changing heading colour to blue
\def\headcolor{\color[rgb]{0, 0, 0.5}}

%% Adding space before section headings
\titlespacing{\section}{0pt}{2ex}{1ex}

%% Entering in name and information
\name{Paul Le Tran}
\info{Office: & \href{https://liberalarts.utexas.edu/economics/ph-d-program/}{Department of Economics},\\
  & College of Liberal Arts,\\
  & The University of Texas at Austin,\\
  & 2225 Speedway,\\
  & BRB 2.128, C3100,\\
  & Austin, Texas 78712.\\
  \textbf{Citizenship}: & \textbf{United States of America}.\\
  Cell: & +1 (512) 704-3025\\
  Email: & \href{mailto:pltran@utexas.edu}{pltran@utexas.edu}\\
  Website: & \href{https://paulletran.com/}{https://paulletran.com/}\\
  GitHub: & \href{https://github.com/PaulTran47}{https://www.github.com/paultran47/}
}
%=================================================

%=================================================
%% Setting up bibliography
\bibliography{tran_paul_le_cv_wps, tran_paul_le_cv_wips, tran_paul_le_cv_pubs}
\DeclareSourcemap{
  \maps[datatype = bibtex, overwrite]{
    \map{
      \step[fieldset = month, null]
    }
  }
}
\ExecuteBibliographyOptions{useprefix = true}

%% Add selected items from .bib files to be shown
\addtocategory{wps}{
  tran_opecnn,
}
\addtocategory{wips}{
  tran_moretimepls,
}
\addtocategory{fin}{
  ct_econthesis,
  tran_mathsthesis,
}
%=================================================

%=================================================
\begin{document}
  \maketitle
  \thispagestyle{firststyle}
  \section{Education}
  ~\begin{tabular}{ll}
    2020--Present; & \textbf{PhD, Economics}, The University of Texas at Austin (UT Austin).\\
    2023; & \textbf{MS en Passant, Economics}, UT Austin.\\
    2017; & \textbf{BA, Mathematics}, Pomona College.\\
    2017; & \textbf{BA, Mathematical Economics}, Pomona College.\\
  \end{tabular}
  \vspace*{0.25em}

  \section{Research}
  \begin{compactitem}\parskip = 0cm
    \item My current interests involve applications of text analysis in macroeconomics, machine learning, and expectations formation. My research interests primarily focus on investigating how text analysis through machine learning can be used in macroeconomics to provide cleaner causal inference.
    \item I have authored or co-authored \numberstringnum{\getrefnumber{sumpapers}} publications on economic topics. A list of these appear below.
  \end{compactitem}
  \vspace*{0.25em}

  %% Adding working papers, works in progress, and publications .bib files
  %% You can inclide \printbib outside of the publications environment. They just won't be counted towards sumpapers
  \printbib{wps}
  \vspace*{-0.75em}
  \printbib{wips}
  \vspace*{-0.75em}
  \begin{publications}
    \printbib{fin}
  \end{publications}
  \vspace*{-0.75em}

  \section{Honours and awards}
  ~\begin{tabular}{lll}
    2020--Present; & \textbf{Graduate Teaching Fellowship}, & UT Austin.\\
    2017; & \textbf{Distinction in Economics Senior Exercise}, & Pomona College.\\
    2014--2015; & \textbf{Pomona College Scholar}, & Pomona College.
  \end{tabular}
  \vspace*{0.25em}

  \section{Teaching history}
  \begin{compactitem}\parskip = 0cm
    \item I earned an Advanced Teaching Preparation Certificate in 2023 from UT Austin.
    \item I have taught master's and undergraduate economics students as a teaching assistant (TA) across 9 semesters.
  \end{compactitem}
  \vspace*{0.70em}
  ~\begin{tabular}{p{3.5cm} p{12cm}}
    Spring 2023; & \textbf{ECO395L: Macro and the Labor Market}, TA, Prof. Mueller.\\
    Fall 2021, Spring 2022, Fall 2022, Spring 2023, Fall 2024; & \textbf{ECO304K: Introduction to Microeconomics} (Synchronous Massive Online Course for fall semesters), TA, Profs. Acchiardo, Geerling, and Mateer, UT Austin.\\
    Summer 2022; & \textbf{ECO325K: Health Economics}, TA, Prof. Schneider, UT Austin.\\
    Fall 2020, Spring 2021; & \textbf{ECO304L: Introduction to Macroeconomics}, TA, Profs. Sadler and Acchiardo, UT Austin.\\
  \end{tabular}
  \vspace*{-0.5em}

  \section{Employment history}
  ~\begin{tabular}{ll}
    2020--Present; & \textbf{Teaching Assistant}, UT Austin.
  \end{tabular}
  \begin{compactitem}\parskip = 0cm
    \item Please see the ``Teaching history'' section for more details.
  \end{compactitem}
  \vspace*{1.25em}

  ~\begin{tabular}{ll}
    2018--2020; & \textbf{Senior Research Assistant}, Board of Governors of the Federal Reserve System.
  \end{tabular}
  \begin{compactitem}\parskip = 0cm
    \item {[Bash, FAME, SAS]} Assisted group of economists tasked with assembling staff's forecasts for U.S. business fixed investment (BFI) ahead of each Federal Open Market Committee (FOMC) meeting.
    \item {[FAME]} Compiled a memo consisting of BFI and business sentiment survey metrics across private sectors and Federal Reserve regional banks, used by Chairman Powell in his June 2019 press conference.
    \item {[Bash, FAME, (P)SQL, SAS]} Calculated relationship between industry-level capital expenditure growth rates from Compustat with metrics of trade exposure and uncertainty with China and other countries. Metrics were published in an internal cross-division FRB memo.
    \item {[FAME]} Implemented tighter incorporation of business sentiment, profit expectations, trade exposure, and uncertainty metrics into models that forecast BFI.
    \item {[Stata]} Calculated trade exposures of industries -- categorised by the U.S. Census Bureau's Manufacturers' Shipments, Inventories, and Orders release -- of final and input goods to China, Europe, and other regions for the purposes of understanding the effects of COVID-19 on supply chains.
    \item {[Bash, Python, (P)SQL]} Created framework that can construct daily, flexible news intensity indices for an arbitrary number of news topics. The framework is capable of performing the dictionary search, outlined in Baker, Bloom, and Davis (2016), across eight million news records in the Thomson Reuters News Archive database and constructing an index in only a few hours.
    \begin{compactitem}
      \item {(Acknowledgement)} The framework was used to measure non-market expectations of tax extensions through text analysis in the following working paper: Chang, Andrew C. (2023). ``\href{https://drive.google.com/file/d/1ipHYM7oGXl9A5VQJqe9VwLM4ZMVJYsC9/view}{\textit{Nothing is Certain Except Death and Taxes: The Lack of Policy Uncertainty from Expiring ``Temporary'' Taxes}}''.
    \end{compactitem}
    \item {[Stata]} Collected and compiled the Drug Enforcement Agency's Automation of Reports and Consolidated Orders Systems database to measure opioid distribution to retail pharmacies as a proxy for opioid usage. Merged with cause-of-death data from the Center for Disease Control and Prevention's National Vitality Statistics Systems database to produce death measurements related to opioid usage.
    \begin{compactitem}
      \item {(Acknowledgement)} The merged database was combined with the Current Population Survey to estimate the causal effects of heroin use on labour market outcomes by proxying for heroin use with prior exposure to oxycodone in the following paper: Cho, David, Daniel I. Garcia, Joshua Montes, and Alison Weingarden (2021). ``\href{https://doi.org/10.17016/FEDS.2021.025}{\textit{Labor Market Effects of the Oxycodone-Heroin Epidemic}}''.
    \end{compactitem}
  \end{compactitem}
  \vspace*{1.25em}
      
  ~\begin{tabular}{ll}
    2017--2018; & \textbf{Research Assistant}, Board of Governors of the Federal Reserve System.
  \end{tabular}
  \begin{compactitem}\parskip = 0cm
    \item {[FAME, R]} Excelled in high-pressure role supporting a group of economists charged with assembling the staff's GDP forecast ahead of each FOMC meeting.
    \item {[FAME, R]} Developed programmes used by FRB staff that expanded the FRB's forecasting apparatus to directly account for measurement error's role in published statistics when determining the true, underlying cyclical position of the economy.
    \item {[FAME]} Created new exhibits highlighting the estimates of the cyclical position of the economy through variables such as GDP, potential output, output gap, \& measurement error in the \href{https://www.federalreserve.gov/monetarypolicy/fomc_historical.htm#tealbooks}{\textit{Tealbook A: ``Economic and Financial Conditions: Current Situation and Outlook''}} and the \href{https://www.federalreserve.gov/monetarypolicy/fomc_historical.htm#greenbooks}{\textit{Greenbooks: ``Current Economic and Financial Condition''}}.
  \end{compactitem}
  \vspace*{0.25em}
      
  \section{Grants}
  ~\begin{tabular}{llll}
    2020--2024; & \textbf{Full PhD Teaching Assistantship Tuition Waiver}, & UT Austin, & \$75,702.\\
    2023; & \textbf{PhD Summer Fellowship}, & UT Austin, & \$3,000.\\
    2016; & \textbf{Harry G. Steele Scholarship}, & Pomona College, & \$4,000.\\
    2013; & \textbf{Flextronics Texas Scholarship}, & Pomona College, & \$1,000.
  \end{tabular}
  \vspace*{0.25em}
    
  \section{Miscellaneous information}
  \begin{compactitem}\parskip = 0cm
    \item \textbf{Programming:} Matlab, Python, Bash, SAS, \href{https://en.wikipedia.org/wiki/FAME_(database)}{FAME}, (P)SQL, R, Stata, EViews, \LaTeX.
    \item \textbf{Web Development:} Vanilla HTML, CSS, JS; Jekyll.
    \item \textbf{Applications:} Visual Studio Code, Emacs, Git, Sublime Text, RStudio, Tableau, Microsoft Office.
    \item \textbf{Operating Systems:} Unix, Linux, Windows.
    \item \textbf{Languages:} Vietnamese (mother tongue), English (mother-tongue level).
  \end{compactitem}
\end{document}
%=================================================

%=================================================
%% Sections that aren't currently displayed, but
%% contain valuable information that could be
%% folded into the CV in the future.

%% Code for detailed ongoing projects
  \iffalse      
  \section{Ongoing research projects}
  ~\begin{tabular}{l p{13cm}}
    2019--Present; & \textit{``How useful is the Evercore ISI Capital Goods Companies Survey in Forecasting Orders and Shipments of Capital Goods?"}, with Eugenio Pinto.
  \end{tabular}
  \begin{compactitem}\parskip = 0cm
    \item Created two VAR models for in-sample and out-of-sample forecasting, with the base model consisting of only total shipments and total new orders, whilst the alternative model included the weekly-frequency business sentiment survey index from Evercore-ISI.
    \item Found our alternative model to perform better at forecasting investment at business cycle turning points, and had smaller forecast errors at most time horizons.
  \end{compactitem}
  \vspace*{1em}
      
  ~\begin{tabular}{l p{13cm}}
    2018--Present; & \textit{``The Story Chain Index: High-frequency Record Updates Linkage Across Time with No Metadata"}, with Andrew C. Chang and Sarah Adler.
  \end{tabular}
  \begin{compactitem}\parskip = 0cm
    \item Developed algorithm that performs general semantic text matching to link high-frequency records across time, in situations where the contents of records are updated, but no meta data to link the records exist.
    \item General semantic text algorithm created from string similarities derived from Levenshtein and Jaquard string distances and record timestamps.
    \item Programmed framework that filtered and chronologically nearly ten million U.S.-centric news item from the linked Thomson Reuters News Archive (TRNA) at the daily frequency.
    \item Constructed a daily flexible news coverage frequency index for any given number of news topics from the linked TRNA.
  \end{compactitem}
  \fi
  
%% Code for coursework
  \iffalse
  \section{Coursework}
  \begin{compactitem}\parskip = 0cm
    \item \textbf{Economics:} Senior Activity in Economics, Senior Seminar in Economics, Advanced Econometrics, Advanced Macroeconomic Analysis, Game Theory for Economists, Econometrics, Corporate Finance,  Managerial Accounting Financial Analysis, Microeconomic Theory,  Macroeconomic Theory, Principles: Microeconomics, Principles: Macroeconomics.
    \item  \textbf{Mathematics:} Senior Thesis in Mathematics, Seminar in Mathematical Exposition, Mathematical Modelling (CP), Statistical Linear Models, Differential Equations/Modelling, Abstract Algebra I: Groups \& Rings, Mathematical Analysis I, Introduction to Analysis, Calculus III (Multi-variable Calculus), Linear Algebra, Economic Statistics, Honours Topics in Calculus II.
  \end{compactitem}
  \fi
  %=================================================